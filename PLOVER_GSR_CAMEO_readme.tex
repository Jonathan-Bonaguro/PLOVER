  \documentclass[11pt, oneside]{article}   	% use "amsart" instead of "article" for AMSLaTeX format
\usepackage{geometry}                		% See geometry.pdf to learn the layout options. There are lots.
\geometry{letterpaper}                   		% ... or a4paper or a5paper or ... 
\usepackage[parfill]{parskip}    		% Activate to begin paragraphs with an empty line rather than an indent
\usepackage{graphicx}				% Use pdf, png, jpg, or eps§ with pdflatex; use eps in DVI mode
								% TeX will automatically convert eps --> pdf in pdflatex
								
\newcommand{\field}[1]{\textsf{\textbf{#1}}}
\newcommand{\code}[1]{\texttt{#1}}
		
\begin{document}
%\section{}
%\subsection{}
\begin{center}
\Large{Documentation for the PLOVER gold standard file \texttt{PLOVER\_GSR\_CAMEO\_readme.pdf}}
\end{center}
This collection of records  is the first attempt to develop gold-standard records for the PLOVER event ontology. It is based on examples extracted 
    from the CAMEO manual\footnote{\texttt{http://eventdata.parusanalytics.com/data.dir/cameo.html}} with considerable editing, since CAMEO 
    coded only actors and events. The records are coded with the version of PLOVER specification that was current as of the \field{dateCoded}
    field: PLOVER is still under development and some of these may change in the future. The file conforms to JSON standards
    and should be readable with any JSON utility.

    It is important to note that the examples in the manual were originally human coded, and from the standpoint of automated 
    coding, they are ideal codings and no known system would be likely to get all of them. There are \field{comment}s are some of
    the more problematic cases. 

    The sentences used in the CAMEO manual were based on actual news stories around the turn of the century, but many have 
    been edited, sometimes multiple times, and do not necessarily refer to actual historical events. The original CAMEO examples have been 
    further edited so that all of the nation-states correspond to entities with names and codes in the current ISO-3166-alpha-3 
    entry in Wikipedia.\footnote{\texttt{https://en.wikipedia.org/wiki/ISO\_3166-1\_alpha-3}} Ethnicity codes correspond to ISO-639-2.
    
    Funding for PLOVER has been provided in part by the U. S. National Science Foundation award SBE-1539302, ``RIDIR: Modernizing 
    Political Event Data for Big Data Social Science Research."
    

\subsection*{Additional comments on the coding}
\begin{enumerate}
\item  Only non-null JSON fields are included. All \field{date} fields are set to `2000-01-01.' The current version does not include
       any \field{mode} or \field{context} fields but these are retained as placeholders.
    
\item City names are resolved as follows:
	\begin{itemize}
	\item Cities in general go to \field{sector} \code{CVL}
	\item  Capital cities go to  \field{sector} \code{GOV} if they are the subject or direct object, otherwise  \field{sector} \code{CVL}
	\end{itemize}
       
\item Some \field{eventLoc} fields have been filled in, usually on the basis of [human-identified] prepositional phrases. In 
       some cases, the actor primary codes were inferred from these locations.

\end{enumerate}
    
\subsection*{Text Markup}

    The \field{textInfo/markup} field shows where the various text fields occur in the sentence. The following markers are 
    used:
      
\begin{table}[htdp]
\begin{center}
\begin{tabular}{ll}
    SRC & source actor and agent \\
    TAR & target actor and agent\\
    SRC/TAR: & Reciprocal source/target in the CAMEO system\\
    EVT & event\\
    LOC & location\\

\end{tabular}
\end{center}
\end{table}%

This markup has been done through a combination of automatic translation from the original \LaTeX ~highlighting in the CAMEO manual
and additional manual editing. It sometimes gets a bit dodgy when there are multiple words in the phrases, but probably is close enough to be useful

\subsection*{Provenance}

Location for this file and the data: \texttt{https://github.com/openeventdata/PLOVER}

The GSR file (and documentation) is licensed under a Creative Commons Attribution-ShareAlike 4.0 International License.

For further information contact: \texttt{schrodt735@gmail.com}

Last update: \today

Copyright \copyright 2016 by the Open Event Data Alliance

\end{document}  